% !TeX root = gentle.tex
% TeX Program=XeLaTeX

\section{Keyboard shortcuts}

Lean uses Unicode symbols that are not on keyboards. The following table shows keyboard shortcuts that are used in VSC for entering the symbols. Enter a space or tab after a shortcut and the symbol will appear. If you hover over a symbol the shortcut will be displayed.

\begin{center}
\begin{tabular}{c|@{\hspace{2em}}l}
\multicolumn{1}{l|@{\hspace{2em}}}{Unicode symbol}&Keyboard shortcut\\
\hline
\Verb+ℕ+   & \Verb+\N+\\
\Verb+ℤ+ & \Verb+\Z+\\
\Verb+ℚ+  & \Verb+\Q+\\
\Verb+ℝ+ & \Verb+\R+\\ 
\Verb+≤+   & \Verb+\le+, \Verb+\leq+\\
\Verb+≥+   & \Verb+\ge+, \Verb+\geq+\\
\Verb+≠+   & \Verb+\ne+, \Verb+\neq+\\
\Verb+←+  & \Verb+\l+\\
\Verb+→+   & \Verb+\r+, \Verb+\imp+\\
\Verb+↔+   & \Verb+\iff+\\
\Verb+∧+   & \Verb+\and+\\
\Verb+∨+   & \Verb+\or+\\
\Verb+∀+   & \Verb+\all+, \Verb+\forall+\\
\Verb+∃+   & \Verb+\ex+, \Verb+\exists+\\
\Verb+¬+   & \Verb+\n+, \Verb+\neq+\\
\Verb+⟨+   & \Verb+\<+\\
\Verb+⟩+   & \Verb+\>+\\
\Verb+∣+   & \Verb+\|+\\
\Verb+⊢+   & \Verb+\|-+, \Verb+\vdash+ \\
\Verb+h₁+ & \Verb+h\1+\\
\Verb+a⁻¹+ & \Verb+a\^-1+,\Verb+a\inv+\\
\Verb+∑+ & \Verb+\sum+\\
\Verb+∏+ & \Verb+\prod+\\
\Verb+α+ & \Verb+\alpha+\\
\end{tabular}
\end{center}

\newpage


\section{Tactics}\label{a.tactics}

\begin{tabular}{p{.16\textwidth}|r|p{.74\textwidth}}
\hline
Tactic & Page & Definition\\
\hline
\Verb+apply+&\pageref{p.apply}&If you have a theorem \Verb+P→Q+ and the goal matches \Verb+Q+ then \Verb+apply P→Q+ removes the goal \Verb+Q+ and adds \Verb+P+ as the new goal.\\
&&If you have a theorem \Verb+Q+ and the goal matches \Verb+Q+ then \Verb+apply Q+ removes the goal and there are no more goals.\\\hline

\Verb+by_contra+&\pageref{p.by-contra}&This tactic removes a goal \Verb+P+, adds the hypothesis \Verb+¬P+ and creates a new goal \Verb+False+.\\\hline

\Verb+calc+&\pageref{p.calc}&A proof that is a sequence of expressions related by equality or inequality together with their proofs.\\\hline

\Verb+contrapose+&\pageref{p.contrapose}&Transforms a goal into its contrapositive. Applied to a hypothesis, it makes the negation of the goal into a hypothesis and the negation of the hypothesis into the goal. 
An exclamation point following \Verb+contrapose+ calls tactic \Verb+push_neg+ on the resulting contrapositive.\\\hline

\Verb+constructor+&\pageref{p.constructor}&Splits a goal into two subgoals: equivalence (\Verb+↔+) into two implications (Modus ponens and Modus ponens reversed) and conjunction (\Verb+∧+) into two conjuncts.\\\hline

\Verb+exact+&\pageref{p.exact}&The tactic \Verb+exact h+ is used when \Verb+h+ is exactly the statement of the current goal, so the goal is now proved. \Verb+exact+ is similar to \Verb+apply+ but limited because it can only be used if the hypothesis exactly matches the goal.\\\hline

\Verb+gcongr+&\pageref{p.gcongr}&\Verb+gcongr+ will attempt to prove the current goal by finding and substituting hypotheses of the form $\mathit{e_1 \; rel\_op \; e_2}$ into the goal.\\\hline

\Verb+have+&\pageref{p.have}&Introduces a new named hypothesis for use in the proof.\\\hline

\Verb+induction'+&\pageref{p.induction}&To prove a goal with \Verb+n+ by induction, specify an induction variable \Verb+k+ and a name \Verb+ih+ for the inductive hypothesis.\\\hline

\Verb+intro+&\pageref{p.intro}&Introduces free variables in place of bound variables in a universally quantified formula. A universally bound variable means that the formula has to hold for an arbitrary value so we simply substitute a variable that is this arbitrary value.\\
&&If the goal is \Verb+P→Q+ then \Verb+intro+ introduces \Verb+P+ as a hypothesis. If \Verb+Q+ is proven then the hypothesis can be \emph{discharged} and \Verb+P→Q+ is proven.\\
&&For an existential goal \Verb+∃ a A(a)+, \Verb+intro c+ introduces \Verb+A(c)+ as the new goal.\\\hline

\Verb+left, right+ &\pageref{p.left-right}& If the goal is a disjunction, tell Lean which disjunction you want to prove.\\\hline

\Verb+linarith+&\pageref{p.linarith}&Solves linear equalities and inequalities.\\\hline

\Verb+norm_num+&\pageref{p.norm-num}&Proves numerical equalities and inequalities that do not use variables.\\\hline

\Verb+range+&\pageref{p.range}&\Verb+range n+ is defined as the set of natural numbers less than $n$.\\\hline

\Verb+rcases+&\pageref{p.rcases}&Given a hypothesis or goal that is a disjunction \Verb+A∨B+ the tactic \Verb+rcases+ splits it into two sub-hypotheses or subgoals \Verb+A+ and \Verb+B+.\\
&&For a disjunctive hypothesis, prove the goal under both sub-hypotheses. For a disjunctive goal, prove one of the subgoals.\\\hline

\Verb+rcases+&\pageref{p.rcases-with}&\Verb+rcases h with ⟨v, h'⟩+ means given a hypothesis \Verb+h+, let \Verb+v+ be a value such that the new goal is \Verb+h'+.\\\hline

\end{tabular}

\begin{tabular}{p{.16\textwidth}|r|p{.74\textwidth}}
\hline

\Verb+rel+&\pageref{p.rel}&Given a hypothesis \Verb+h+ which is a relation $\mathit{e_1 \; rel\_op \; e_2}$, \Verb+rel [h]+ will attempt to prove the current goal by substituting \Verb+h+ into the goal.\\\hline

\Verb+rfl+&\pageref{p.rfl}&\Verb+rfl+ solves a goal that is an equality if they are \emph{definitionally equivalent}, that is, they define the same entity.\\\hline

\Verb+ring+&\pageref{p.ring}&Proves equalities that can be proved directly from the axioms of a commutative ring without taking any hypotheses into account.\\\hline

\Verb+rintro+&\pageref{p.rintro}&Performs \Verb+intro+ and then \Verb+rcases+ to split the resulting hypothesis.\\\hline

\Verb+rw+&\pageref{p.rw}&\Verb+rw [eqn]+, where \Verb+eqn+ is an equation or an equivalence, rewrites the goal by replacing occurrences of the the left-hand side of \Verb+eqn+ with the right-hand side.\\
&&\Verb+rw [← eqn]+ rewrites the goal by replacing occurrences of the right-hand side of \Verb+eqn+ with the left-hand side.\\
&&\Verb+rw [eqn] h+ and \Verb+rw [← eqn] h+ rewrite expressions in the hypothesis \Verb+h+.\\\hline

\Verb+simp+&\pageref{p.simp}&The \Verb+simp+ tactic uses lemmas and hypotheses in \Verb+Mathlib+ to simplify a goal or hypothesis.\\\hline

\Verb+sorry+&\pageref{p.sorry}&Proves any theorem.\\\hline

\Verb+symm+&\pageref{p.symm}&Replace an equality \Verb+t = u+ by \Verb+u = t+.\\\hline

\Verb+twoStep-+ \Verb+Induction+&\pageref{p.two-step}&Proof by induction when there are two base cases and two inductive hypotheses.\\\hline

\Verb+use+&\pageref{p.use}&Given an existential goal \Verb+∃ c A(c)+, this tactic introduces a free variable for the bound variable \Verb+c+ to form a new goal. It will also attempt to prove the goal using the hypotheses.\\\hline

\end{tabular}

\paragraph{Comparison of numerical tactics} The follow table and examples clarify the differences between the various tactics for proving numerical equalities and inequalities.
\begin{center}
\begin{tabular}{l|c|c|c}
\hline
Tactic & Inequalities & Use hypotheses & Use variables\\\hline
\Verb+linarith+ &yes& yes & yes\\
\Verb+norm_num+ & yes &yes & no\\
\Verb+ring+ & no & no & no \\
\end{tabular}
\end{center}

\bigskip

\begin{Verbatim}[numbers=none]
example : 1 + 3 = 4 := by ring
example : 1 + 3 = 4 := by linarith
example : 1 + 3 = 4 := by norm_num

example : 1 + 3 < 5 := by ring      -- Error
example : 1 + 3 < 5 := by linarith
example : 1 + 3 < 5 := by norm_num

example (h : k = 3): 1 + k = 4 := by ring      -- Error
example (h : k = 3): 1 + k = 4 := by linarith
example (h : k = 3): 1 + k = 4 := by norm_num  -- Error
\end{Verbatim}

\newpage

\section{Tips}

\begin{tabular}{p{.16\textwidth}|r|p{.74\textwidth}}
\hline
Topic & Page & Tip\\
\hline
Anonymous&\pageref{p.this}&A hypothesis can be unamed and can be subsequently accessed as \Verb+this+.\\
hypothesis&\\\hline

Associativity&\pageref{p.associativity}&The default associativity in Lean is \emph{left}: \verb!a + b + c! means \verb!(a + b) + c!. Even if the operation is associative, you still have to prove associativity for a given expression.\\\hline

\Verb+class+&\pageref{p.class}&A class is a function that takes one or more types of elements and returns a type which can be used to create objects of that type.\\\hline

Division&\pageref{p.division}&The division operator in Lean is not the \Verb+|+ symbol on your keyboard, but a similar Unicode symbol \Verb+∣+ obtained by typing \Verb+\|+.\\\hline

\Verb+done+&\pageref{p.done}&All proofs should be terminated by \Verb+done+. This is not necessary but if your proof is not complete, \Verb+done+ will display a message.\\\hline

Don't care&\pageref{p.dont-care}&When a name or value is syntactically required but you don't care what its value is, you can use the underscore symbol \Verb+_+ instead.\\\hline

\Verb+example+&\pageref{p.example}&\Verb+example+ can be used to state an unnamed theorem.\\\hline

\Verb+Finset+, \Verb+BigOperators+&\pageref{p.finset}&Given a finite set of numbers \Verb+s+ and a function \Verb+f+ over these numbers, \Verb+Finset+ defines the expressions \Verb+∑ x in s f x+ and \Verb+∏ x in s f x+, where \Verb+BigOperators+ enables the use of  \Verb+∑+ and \Verb+∏+.\\\hline

Indentation&\pageref{p.indentation}&All statements used to prove a hypothesis must be uniformly indented.\\\hline

Infoview&\pageref{p.infoview}&Enter \Verb+ctrl-shift-enter+ to open the Lean Infoview where hypotheses, goals and errors are displayed. I have found it helpful to place the cursor just before the source line, and then to alternate between \Verb+Home+ and \Verb+End+ while looking at Infoview.\\\hline

Implicit&\pageref{p.implicit}&Parameters declared with braces, such as \Verb+{a b c : Int}+ state that \Verb+a b c+ are \emph{implicit} bound variables in the theorem whose actual names and types can be inferred when the theorem is used.\\\hline

Multiple tactics& \pageref{p.multiple}&A sequence of tactics can be written on one line with semicolons between them.\\\hline

Negation&\pageref{p.negation}&A negation \Verb+¬p+ is defined as \Verb+p → False+ so it is proved by introducing \Verb+p+ as a hypothesis and then showing that this results in a contradiction by proving \Verb+False+.\\\hline

\Verb+range+&\pageref{p.range}&\Verb+range n+ is defined as the set of natural numbers less than \Verb+n+.\\\hline

Right-to-left&\pageref{p.right-to-left}&The symbol \Verb+<|+ means to compute the expression to its right and pass it to the left.\\\hline

\Verb+structure+&\pageref{p.structure}&A structure is a collection of data together with a set of constraints on that data.\\\hline

Tactic&\pageref{p.tactic}&If you hover over a tactic its specification will be displayed.\\\hline

Theorem&\pageref{p.theorem}&If you hover over a theorem the stattement of the theorem will be displayed.\\\hline

Types&\pageref{p.types}&By convention types are denoted by Greek letters \Verb+α+\\\hline
\end{tabular}

