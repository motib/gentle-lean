% !TeX root = gentle.tex
% !TeX Program=XeLaTeX

\section{Geometry}

We learn in school that Euclid was able to give an axiomatization of geometry using only five axioms. A more careful investigation showed that Euclid's axioms are not complete. There are now modern axiomatizations, in particular, those given by David Hilbert in 1899 and by Alfred Tarski in 1926. A recent book gives an axiomatization of geometry suitable for students: David M. Clark and Samrat Pathania, \textit{A Full Axiomatic Development of High School Geometry}, Springer, 2023. Here I formalize the first few steps in the axiomatization and I invite the readers to expand it.\footnote{Additional definitions and axioms are given in the Lean source file, but no further theorems are proved.}

A (two-dimensional) \emph{point} is defined as a structure with two elements of type real. A \emph{Cartesian plane} is defined as a set of points together with a function !d! which maps two points to a non-negative real number. This function represents the distance between two points in the plane and must fulfill two conditions: the distance of a point from itself is zero and the function is symmetrical.
\begin{Verbatim}
structure Point where
  x : ℝ
  y : ℝ

structure Cartesian_Plane where
  P : Set Point
  d : Point → Point → NNReal
  zero : ∀ A : Point, d A A = 0
  symm : ∀ A B : Point, d A B = d B A
\end{Verbatim}

\boxed{Tip \Verb+structure+}{structure}{A structure is a collection of data together with a set of constraints on that data.}

Now that we have characterized what a plane is, we can declare one.

\begin{Verbatim}[firstnumber=last]
namespace Cartesian_Plane
variable (Plane : Cartesian_Plane)
\end{Verbatim}

Three constructs are now defined; the definitions are in line with our intuitive understanding of Euclidean geometry.

The point $B$ is \emph{between} $A$ and $C$ if $d(A,C)=d(A,B)+d(B,C)$.
\begin{Verbatim}[firstnumber=last]
def btw (A B C : Point) := Plane.d A C = Plane.d A B + Plane.d B C
\end{Verbatim}

A \emph{line} $AB$ consists of two points $A$ and $B$, together with all points between them, all points further from $A$ than $B$, and all points further from $B$ than $A$. Similar definitions can be given for a segment and a ray.

\begin{Verbatim}[firstnumber=last]
def line (A B : Point) := {A, B} ∪ 
  {D : Point | Plane.btw A D B ∨ Plane.btw A B D ∨ Plane.btw D A B}
\end{Verbatim}

Three points are \emph{collinear} if they are on the same line.

\begin{Verbatim}[firstnumber=last]
def col (A B C : Point) : Prop := C ∈ Plane.line A B
\end{Verbatim}

The first lemma in the book is called $SYM$ and states that if $B$ is between $A$ and $C$ then the three points are collinear and $B$ is between $C$ and $A$. We prove the two parts separately.

To prove that the points are collinear, we introduce the hypothesis and use the definitions of !col! and !line!. The goal becomes\\
\indnt{}\Verb+{A, B} ∪ {D : Point |+\\
\indnt{}\indnt{}\Verb+ Plane.btw A D B ∨ Plane.btw A B D ∨ Plane.btw D A B}+\\
This is a disjunction with four disjunctions and we only need to prove one of them. The second one results in a goal that is the same as the hypothesis.\footnote{Recall that operators associate to the right so \Verb+right ; left+ does not work.}
\begin{Verbatim}[firstnumber=last]
lemma L_1_col (A B C : Point) : 
    Plane.btw A B C → Plane.col A B C := by
  intro h
  rw [col, line]
  right ; right ; left
  exact h
  done
\end{Verbatim}

To prove the symmetry of !btw!, we first introduce the hypothesis and use the definition of !btw! in terms of !d!. \\
\indnt{}!d(C,A) = d(C,B) + d(B,A)!\\
The symmetry constraint of d and the commutativity of addition are used to re-arrange the equation. The result is\\
\indnt{}!d(A,C) = d(A,B) + d(B,C)!\\
which is replaced by the definition of !btw!.
\begin{Verbatim}[firstnumber=last]
lemma L_1_btw (A B C : Point) :
    Plane.btw A B C → Plane.btw C B A := by
  intro h
  rw [btw]
  rw [Plane.symm B A, Plane.symm C B, Plane.symm C A, add_comm]
  rw [← btw]
  exact h
  done
\end{Verbatim}

