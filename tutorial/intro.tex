% !TeX root = gentle.tex
% !TeX Program=XeLaTeX

\section{Introduction}

Lean 4 is a \emph{proof assistant}. You enter your proof into Lean in a formal language and the system checks the correctness of the proof. It displays the current set of hypotheses and goals, and it is capable of performing many simple proofs automatically. This tutorial is, as its name states, a gentle introduction to Lean intended for students and others who have no previous experience with a proof assistant, though a basic knowledge of propositional and first-order logic is assumed.

Constructs of Lean are explained and demonstrated within complete proofs of theorems of arithmetic and logic. The source code is heavily commented and additional explanations are given. As you gain more experience with Lean, fewer comments are given. Proof tactics are introduced one at a time as needed and only standard tactics are used. Tables of keyboard shortcuts, tactics and tips are given in the appendices.

The Lean source code of the proofs and the \XeLaTeX{} source of this document can be found at

\indnt{}\url{https://github.com/motib/gentle-lean}.

\subsection*{Installation}

To install Lean, follow the instructions for your operating system at\\
\indnt\url{https://leanprover-community.github.io/get_started.html}.

The Lean community uses \emph{Visual Studio Code (VSC)}\\
\indnt{}\url{https://code.visualstudio.com/}\\
It is a versatile environment with lots of features, so be sure to study VSC tutorials before starting to work with Lean.

You must work within a project framework as described in\\ \indnt\url{https://leanprover-community.github.io/install/project.html}.\\ Once you have created a project, open the project folder and create new source files with the extension !lean!.

\subsection*{Syntax}

Lean uses two types of comments: 
\begin{itemize}
\item Line comments start with !--! and continue to end of the line.
\item Range comments start with !/-! and continue to !-/!.
\end{itemize}

The convention is to use spaces around each operator:\\
\indnt{}!a ≤ b ∧ b ≤ a → a = b!\\
and I will do so in this tutorial. You may ignore this convention in your proofs.

There is also a convention for naming theorems: the name is written in lower case with underscores between the parts of the name. The name specifies the meaning of the theorem in a few words or abbreviations, for example, !min_le_right! is !min a b ≤ b! and !min_le_left! is !min a b ≤ a!.

\subsection*{Tips for working with Lean}

\boxed{Tip: Infoview}{infoview}{Enter \Verb+ctrl-shift-enter+ to open the Lean Infoview where hypotheses, goals and errors are displayed. To understand the effect of applying a tactic, place the cursor just before the source line and then alternate \Verb+Home+ and \Verb+End+ while looking at Infoview.}

\boxed{Tip: tactic}{tactic}{If you hover over a tactic its specification will be displayed.}

\boxed{Tip: theorem}{theorem}{If you hover over a theorem the stattement of the theorem will be displayed.}

Theorems are defined to be widely applicable, so initially you will not be able to understand the full statement of a theorem in the library. In this tutorial only natural numbers and integers are used, so you can interpret the theorems narrowly, for example,\\
\indnt{}!dvd_mul_right (a b : Nat) : a ∣ a * b!

\subsection*{Preamble}

The Lean community has developed a mathematical library called !Mathlib!. You will work with the definitions and theorems defined there, which are imported in the first lines of your Lean source file. Preambles do not appear in the examples here but can be found in the source files.

\begin{enumerate}
\item The library of proof tactics:\\
\indnt{}!import Mathlib.Tactic!

\item Other libraries contain the definitions and theorems of data types:\\
\indnt{}!import Mathlib.Data.Nat.Prime!

\item !namespace! ensures that the names you give do not clash with the names of theorems in !Mathlib!. The examples in this tutorial use the same names as those in \Verb+Mathlib+ so be sure to retain the !namespace!.\\ \indnt{}!namespace gentle!\\
Alternatively, you can modify the names of the theorems, for example, replace !gcd_comm! by !my_gcd_comm!.\\

\item You can open a namespace so that you do not have to write a full name:  !factorial! instead of !Nat.factorial!:\\
\indnt{}!open Nat!
\end{enumerate}

\newpage

\subsection*{References}
\textbf{Lean resources}
\begin{itemize}
\item J. Avigad and P. Massot. \textit{Mathematics in Lean}.\\
\indnt\url{https://leanprover-community.github.io/mathematics_in_lean/}\\
This is a comprehensive presentation on the use of Lean to prove mathematical theorems. Most of the proofs in this tutorial are based on examples given there.
\item The website of the Lean Community contains links to important resources and to the Zulip chat where you can ask questions.\\
\indnt{}\url{https://leanprover-community.github.io/}
\item J. Avigad, L. de Moura, S. Kong and S. Ullrich. 
\textit{Theorem Proving in Lean 4}. \\
\indnt{}\url{https://lean-lang.org/theorem_proving_in_lean4/}\\
A formal presentation of Lean.
\end{itemize}
\textbf{Learning materials on Lean}
\begin{itemize}
\item The games on the Lean Game Server are a fun way to start learning Lean.\\
\indnt{}\url{https://adam.math.hhu.de/}
\item D. J. Velleman. \textit{How to Prove It with Lean}.\\
\indnt{}\url{https://djvelleman.github.io/HTPIwL/}\\
This is a textbook on methods of proof in mathematics. To simplify learning it uses tactics developed by the author.
\item H. Macbeth. \textit{The Mechanics of Proof}.\\
\indnt{}\url{https://hrmacbeth.github.io/math2001/}\\
This is a textbook on methods of proof in mathematics. To simplify learning it uses tactics developed by the author.
\end{itemize}
\textbf{Mathematical logic}
\begin{itemize}
\item M. Ben-Ari. \textit{Mathematical Logic for Computer Science (Third Edition)}, Springer, 2012.\\
An introduction to mathematical logic including both syntactical and semantic proof methods, as well as sections on temporal logic and program verification.
\item M. Huth and M. Ryan. \textit{Logic in Computer Science: Modelling and Reasoning about Systems (Second Edition)}, Cambridge University Press, 2004.\\
An introduction to mathematical logic with emphasis on natural deduction, temporal logic and model checking.
\end{itemize}

\subsection*{Acknowledgment}

I am deeply indebted to the members of the Lean Community for their patience and help as I took my first steps in Lean.

