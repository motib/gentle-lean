% !TeX root = gentle.tex
% !TeX Program=XeLaTeX

\section{Introduction}

Lean 4 is a \emph{proof assistant}. You enter your proof into Lean in a formal language and the system checks the correctness of the proof. It displays the current set of hypotheses and goals, and it is capable of performing many simple proofs automatically. This tutorial is, as its name states, a gentle introduction to Lean intended for students and others who have no previous experience with a proof assistant.

Constructs of Lean are explained and demonstrated within complete proofs of theorems of arithmetic and logic. The source code is heavily commented and additional explanations are given. As you gain more experience with Lean, the number of comments on a proof is reduced. Proof tactics are introduced one at a time as needed as are tips. Tables of keyboard shortcuts, tactics and tips are given in the appendices. The tips include information on definitions and notations used by Lean.

This tutorial limits itself to proving properties of integers and natural numbers, for example, theorems on greatest common denominators and prime numbers. Beyond that you need a basic knowledge of propositional and first-order logic as found in introductory textbooks.\footnote{Only tactics are used in the proofs and they are applied one at a time.}

The Lean source code of these proofs and the \XeLaTeX{} source of this document can be found at
\indnt{}\url{https://github.com/motib/gentle-lean}.

\subsection*{Installation}

To install Lean, follow the instructions for your operating system at\\
\indnt\url{https://leanprover-community.github.io/get_started.html}.

The Lean community uses \emph{Visual Studio Code (VSC)} \url{https://code.visualstudio.com/}. It is a very versatile environment with lots of features, so be sure to study VSC tutorials before starting to work with Lean.

You \emph{must} work within a project framework as described in\\ \indnt\url{https://leanprover-community.github.io/install/project.html}.\\ When you start Lean to work on an existing project, you must open the \emph{Folder} containing the project. Once you have created a project, you can create new source files which must have the extension !lean!.

\subsection*{Tips for working with Lean}

\boxed{Tip: Infoview}{infoview}{Enter \Verb+ctrl-shift-enter+ to open the Lean Infoview where hypotheses, goals and errors are displayed. To understand the effect of applying a tactic, I have found it helpful to place the cursor just before the source line, and then to alternate between \Verb+Home+ and \Verb+End+ while looking at the Infoview.}

\boxed{Tip: tactic}{tactic}{If you hover over a tactic its specification will be displayed.}

For example, the specification of the tactic !exact! is\\
\indnt{}!exact e! closes the main goal if its target type matches that of !e!.

\boxed{Tip: theorem}{theorem}{You can display the statement of a theorem by hovering over its name.}

Theorems are defined to be widely applicable, so initially you will not be able to understand the full statement of a theorem. In this tutorial only natural numbers and integers are used, so you can interpret the theorems narrowly, for example,\\
\indnt{}!dvd_mul_right (a b : Nat) : a ∣ a * b!

\subsection*{Preamble of a Lean program}

The Lean community has developed an extremely large mathematical library called !Mathlib!. You will work with the definitions and theorems defined there, which must be imported as the first lines in you Lean source file.

\indnt{}!import Mathlib.Tactic!\\
imports the proof tactics.

You will need least one other library that contains the definitions and theorems of the data type that you want to work on, for example,\\
\indnt{}!import Mathlib.Data.Nat.Prime!

The examples here are enclosed in\\
\indnt{}!namespace gentle!\\
to ensure that their names do not clash with the names of theorems in !MathLib!. In addition,\\
\indnt{}!open Nat!\\
will often be used to open the namespace of !Nat! so that we can write !factorial! instead of !Nat.factorial!.

Preamble do not appear in the examples here but can be found in the Lean source files.

\subsection*{Syntax}

Lean uses two types of comments: 
\begin{itemize}
\item Line comments start with !--! and continue to end of the line.
\item Range comments start with !/-! and continue to !-/!.
\end{itemize}

Within Mathlib the convention is to use spaces around each operator:\\
\indnt{}!a ≤ b ∧ b ≤ a → a = b!\\
and I will do so in this tutorial. You may ignore this convention in your proofs: !a≤b∧b≤a→a=b!.

There is also a convention for naming theorems: the name is written in lower case with underscores between the parts of the name. The name specifies the meaning of the theorem in a few words or abbreviations, for example, !min_le_right! is !min a b ≤ b! and !min_le_left! is !min a b ≤ a!.

\newpage

\subsection*{References}
\textbf{Mathematical logic}
\begin{itemize}
\item M. Ben-Ari. \textit{Mathematical Logic for Computer Science (Third Edition}, Springer, 2012.\\
An introduction to mathematical logic including both syntactical and semantic proof methods, as well as sections on temporal logic and program verification.
\item M. Huth and M. Ryan. \textit{Logic in Computer Science: Modelling and Reasoning about Systems (Second Edition)}, Cambridge University Press, 2004.\\
An introduction to mathematical logic with emphasis on natural deduction, temporal logic and model checking.
\end{itemize}
\textbf{Lean resources}
\begin{itemize}
\item J. Avigad and P. Massot. \textit{Mathematics in Lean}.\\
\indnt\url{https://leanprover-community.github.io/mathematics_in_lean/}\\
This is a comprehensive presentation of the use of Lean to prove mathematical theorems. The proofs in this document are based on examples given there.
\item J. Avigad, L. de Moura, S. Kong and S. Ullrich. 
\textit{Theorem Proving in Lean 4}. \\
\indnt{}\url{https://lean-lang.org/theorem_proving_in_lean4/}\\
A formal presentation of Lean.
\item The website of the Lean Community contains links to important resources and to the Zulip chat where you can ask questions.\\
\indnt{}\url{https://leanprover-community.github.io/}
\end{itemize}
\textbf{Learning materials on Lean}
\begin{itemize}
\item The games on the Lean Game Server are a fun way to start learning Lean.\\
\indnt{}\url{https://adam.math.hhu.de/}
\item D. J. Velleman. \textit{How to Prove It with Lean}.\\
\indnt{}\url{https://djvelleman.github.io/HTPIwL/}\\
This work focuses on proof techniques in mathematics; to simplify learning it uses tactics developed by the author.
\item H. Macbeth. \textit{The Mechanics of Proof}.\\
\indnt{}\url{https://hrmacbeth.github.io/math2001/}\\
This work focuses on proof techniques in mathematics; to simplify learning it uses tactics developed by the author.
\end{itemize}

\subsection*{Acknowledgment}

I am indebted to the members of the Lean Community for their patience and help as I took my first steps in Lean.
